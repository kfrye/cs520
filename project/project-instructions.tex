\documentclass{article}

\usepackage{xstring}
\usepackage{fancyhdr}
\usepackage{extramarks}
\usepackage[plain]{algorithm}
\usepackage{algpseudocode}
\usepackage{parskip}
\usepackage{enumerate}
\usepackage{setspace}
\usepackage{subfig}
\usepackage{listings}
\usepackage{pgfplots}
\usepackage{url}
\usepackage[hidelinks]{hyperref}
\usepackage{color}
\usepackage{courier}
\usepackage{hyperref}
\usepackage{graphicx}

%
% Basic Document Settings
%
\graphicspath{ {images/} }

\topmargin=-0.45in
\evensidemargin=0in
\oddsidemargin=0in
\textwidth=6.5in
\textheight=9.0in
\headsep=0.25in

\linespread{1.1}

\pagestyle{fancy}
\lhead{CreateGraphics Tutorial}
\rhead{Grace}
\lfoot{\lastxmark}
\cfoot{\thepage}

\renewcommand\headrulewidth{0.4pt}
\renewcommand\footrulewidth{0.4pt}

\setlength\parindent{0pt}

\definecolor{dkgreen}{rgb}{0,0.6,0}
\definecolor{gray}{rgb}{0.5,0.5,0.5}
\definecolor{mauve}{rgb}{0.58,0,0.82}

\lstset{frame=tb,
  language=Java,
  aboveskip=3mm,
  belowskip=3mm,
  showstringspaces=false,
  columns=flexible,
  basicstyle={\small\ttfamily},
  numbers=none,
  numberstyle=\tiny\color{gray},
  keywordstyle=\color{blue},
  commentstyle=\color{dkgreen},
  stringstyle=\color{mauve},
  breaklines=true,
  breakatwhitespace=true,
  tabsize=3
}

\begin{document}

\section{Introduction}
The CreateGraphics package allows the developer to:

\begin{itemize}
\item Generate basic graphics
\item Add event listeners
\item Sound functionality for built-in mp3 files
\item Basic "tween" animations
\end{itemize}

Functionality is performed using the createjs javascript library available at:\\
\url{http://www.createjs.com/}. 

\section{Graphics}
In order to work with the CreateGraphics package, you need to include the 
following line at the top of your Grace file:

\texttt{import createGraphics as cg}

The graphics object needs to be created with the following command:

\texttt{def graphics = createGraphics(width,height)} 

where width and height correspond to the desired graphics window width
and height. For example:

\begin{lstlisting}
def width = 300
def height = 300
def graphics = cg.createGraphics(width, height)
\end{lstlisting}

\section{Shapes}
The following shape objects are available in CreateGraphics:Circle, Rectangle, Rounded Rectangle, PolyStar, Ellipse, and Text.
To draw one of these objects to the screen, pass a message to the graphics object:

\begin{itemize}
\item addCircle
\item addRect
\item addPolyStar
\item addRoundRect
\item addEllipse
\end{itemize}

For instance, to add a circle to the window, you might do the following:
\begin{lstlisting}
import createGraphics as cg
def graphics = cg.createGraphics(300, 300)
graphics.addCircle
graphics.draw
\end{lstlisting}
Each shape has different parameters that are used to create it. These parameters have default values, so you don't need to set each
one every time you create an shape.

\subsection{Common Parameters}
There are a few parameters that are common to each type of shape.

\begin{itemize}
\item \textbf{location} (Point): The x,y coordinates where the shape will be placed in the graphics window. Coordinates are expressed in Grace "Point" notation: x@y. Keep in mind that the origin is in the upper left corner of the window, so 10@10 will be 10 down and 10 right from the corner of the window.
\item \textbf{color} (String): The color of the shape. Most basic colors can be set as "red", "blue", etc. However, you can also use 6-digit hex numbers 
such as "\#CC3300" that corresponds to an HTML 5 hex colors. See \url{http://www.w3schools.com/tags/ref_colorpicker.asp} for more details.
\item \textbf{fill} (Boolean): Whether or not you want to fill in the shape when it is drawn on the window.
\end{itemize}

\subsection{Circle}
\textbf{Create:} graphics.addCircle

Parameters:
\begin{itemize}
\item \textbf{radius} (Number): The length of the circle radius
\end{itemize}

\subsection{Rectangle}
\textbf{Create:} graphics.addRect

Parameters:
\begin{itemize}
\item \textbf{width} (Number): Width of the rectangle
\item \textbf{height}(Number): Height of the rectangle
\end{itemize}

\subsection{Rounded Rectangle}
\textbf{Create:} graphics.addRoundRect

Parameters:
\begin{itemize}
\item \textbf{width} (Number): Width of the rectangle
\item \textbf{height}(Number): Height of the rectangle
\item \textbf{radius} (Number): Radius of the rounded corners
\end{itemize}

\subsection{PolyStar}
\textbf{Create:} graphics.addPolyStar

Parameters:
\begin{itemize}
\item \textbf{size} (Number): Length of each side of the star
\item \textbf{sides} (Number): Number of sides
\item \textbf{pointSize} (Number): Size of the points
\item \textbf{angle} (Number): Angle between the points
\end{itemize}

\subsection{Ellipse}
\textbf{Create:} graphics.addEllipse

Parameters:
\begin{itemize}
\item \textbf{width} (Number): Width of the ellipse
\item \textbf{height}(Number): Height of the ellipse
\end{itemize}

\subsection{Text}
\textbf{Create:} graphics.addText

Parameters:
\begin{itemize}
\item \textbf{content} (String): The content of the string
\end{itemize}

\section{Drawing a Shape}
To draw a shape on the graphics window, first create it, then configure it, and then draw it. The following code
creates the output down in Figure \ref{fig:red_circle}.
\begin{lstlisting}
import createGraphics as cg
def graphics = cg.createGraphics(200, 200)
def circle = graphics.addCircle
circle.color := "red"
circle.radius := 20
circle.position := 30@30
circle.fill := true
graphics.draw
\end{lstlisting}

\begin{figure}[h]
\includegraphics{red_circle}
\centering
\caption{Creating a red circle}
\label{fig:red_circle}
\end{figure}

\section{Adding a Click Handler}
Adding a click handler to a shape defines a block of code that will be executed when the shape is clicked. For instance,
let's say that we want the red circle to turn blue when it is clicked, and we want to add a message to the user. 
Then you would add something like this:
\begin{lstlisting}
import createGraphics as cg
def graphics = cg.createGraphics(200, 200)
def circle = graphics.addCircle
circle.color := "red"
circle.click := { 
  print("clicked circle") 
  circle.color := "blue"
  graphics.draw
}
graphics.draw
\end{lstlisting}
Notice that "graphics.draw" is located twice in this code. The second instance of graphics.draw (located in the last line
of the code block) initially draws the circle onto the window. The first instance of graphics.draw (located at the end of the circle.click block),
updates the circle on the window after it has been clicked. This instance of graphics.draw will occur at some point in the future when
the circle is clicked.

\section{Adding sound}
CreateGraphics supports basic sounds. All sounds are preloaded in the browser and cannot be customized at this time.
To play a sound, just use the "play" method of the graphics object. For example:
\begin{lstlisting}
import createGraphics as cg
def graphics = cg.createGraphics(200, 200)
def circle = graphics.addCircle
circle.color := "red"
circle.click := { 
  print("clicked circle") 
  graphics.play("bicycle_bell")
  graphics.draw
}
graphics.draw
\end{lstlisting}


The following sounds are available: note1, note2, note3, note4, note5, note6, note7, note8, bicycle\_bell, snap, 
whoosh, shutter.
\end{document}
